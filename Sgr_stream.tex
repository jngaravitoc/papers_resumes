\documentclass[12pt]{article}
\usepackage[hmargin=2.0cm,vmargin=1cm]{geometry}
\usepackage[utf8]{inputenc}
\usepackage{graphicx}
\usepackage{float}
\usepackage{cite}
\usepackage{natbib}
\usepackage{amsmath}

\title{\begin{LARGE}
{Modelling the Sagittarius stream}
\end{LARGE}}
\begin{document}
\maketitle

Some characteristics

\begin{itemize}
\item First observed by Ibata 94
\item Large population of young relatively metal rich Sgr M-giant wrapping 360
across the sky  (Majewski 03)
\item debris streamer continues through the North Galactic Pole passes over the 
solar neighborhood toward the galactic anitcenter. (Belokurov 06)
\item Evolution in metallicity distribution function.
\end{itemize}

Efforts trying to modelled the stream.

\begin{itemize}
\item Johnston 95, 99
\item Velazquez \& White 95
\item Edlesohn \& Elmegreen 97
\item Ibata et al 07
\item Gomez-Flechoso et al 99
\item Helmi \& White 2001
\item Martinez-Delgado et al 2004
\end{itemize}

MW DM halo:

\begin{itemize}
\item Ibata et al 01 (Spherical)
\item Helmi 04 (Prolate)
\item Johnston et al 05 (Oblate halo)
\item Law et al 05
\item Fellhauer et al 06 (Spherical)
\item Martinez Delgado et al 07 (Oblate)
\item Law, Majewski, Johnston 09 ()
\item Law \& Majewski (Triaxial)
\end{itemize}

\textbf{Goal:Reproduce the angular position and distances, radial velocities of 
the tidal debrin in the Sgr leadin arm } 


\end{document}
